\documentclass[12pt]{article}
\usepackage{amsmath}
\usepackage{graphicx}
\usepackage{hyperref}
\usepackage[latin1]{inputenc}
\usepackage{epstopdf} %%package to overcome problem with eps in pdf files

\title{Software Engineering in Practice. My First Open Source Contribution}
\author{Ioannis Petropoulos - t8160107@aueb.gr}
\date{20/05/19}

\begin{document}
\maketitle

% Here is the abstract.
\begin{abstract}
  The most important objective of the Software Engineering in Practice course was to familiarize with the real conditions of open source software development and to encourage the technologies used for this matter. In the meanwhile, that was a unique opportunity to co-operate with engineers that maintain the whole project through a Github repository and be a part of the development team.  
	\end{abstract}
%---

\section*{Introduction}
   This course has been a springboard for my academic career. Under the supervision of professor Diomidis Spinellis and the guidence of PhD candidate Antonis Gkortzis, I managed to gather knowledge and gain experience that will make up a solid foundation for my professional evolution in the field of Software Engineering.

\section{Project understanding}

\section{Kappa}
  test
  
\begin{figure}[tph!]
\leftline{\includegraphics[totalheight=6cm]{favicon.png}}
    \caption{used by}
    \label{fig:verticalcell}
\end{figure}

\subsection{Something}

    Width of changes
    Quality of implementation
    Completion
    Check
    Working with development team
    Presentations
    Deliverable quality
    Organization in GitHub (wiki, issues, commits, branching)
    Reviewing Code Entries - code reviews
  
\end{document}